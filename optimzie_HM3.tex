\documentclass{ctexart}
\usepackage[]{amsmath}
\usepackage[]{graphicx}
\usepackage[]{algorithm}
\usepackage[]{algorithmicx}
\usepackage[]{algpseudocode} 
\begin{document}
\title{最优化第三次上机作业}
\author{王恒亮 \quad 学号:1601210102}
\date{}
\maketitle
本次实现了解决Minimax Problems的三种方法,分别为Least pth Method\cite{Charalambous1979}, Smoothing Method\cite{Xu2001}和Adaptive Smoothing Method\cite{Polak2003}。以下分别介绍算法及程序实现,实验结果及分析。
\section{算法及程序实现}
\subsection{Least pth Method}
Least pth Method 主要通过优化U函数来解决Minimax问题,U函数的定义如下:
\begin{equation}
	\label{eq:ufun}
	U(x,u,p,\xi)=\left\{
		\begin{align}
			M(x,\xi)(\sum_{i\in S(x,\xi)}{u_i(\frac{f_i(x)-\xi}{M(x,\xi)})^q})^{1/q}, for M(x,\xi) \neq 0,\\
		0, for M(x,\xi) = 0
		\end{align}\right.
\end{equation}
其中:
\begin{align}
	u_i \geq 0 \text{and} \sum_{i\in I} = 1\\
	M(x, \xi) = \max_{i\in I}{(f_i(x) - \xi)}
\end{align}
算法的框架如下:
\begin{description}
\item 取$p>1, \xi^{(1)} =0,u_i^{(1)}=1,i\in I,x=x^0,r=1$
\item 固定$u,p,\xi$,在公式\ref{eq:ufun}中对x取极小,最优解设为$x^r$
\item 更新$u$
	\[u^{r+1}_i\frac{v_i^{r+1}}{\sum_{j\in I}{v_j^{r+1}}}, i\in I\]
	其中:
	\[v_{i}^{r+1}=\left\{
		\begin{align}
			u_i^r(\frac{f_i(x^r)-\xi}{M(x^r,\xi)})^{q-1}, for i \in S(x^r,\xi)\\
			0 for i\in I - S(x^r, \xi)
		\end{align}\right.\]
	\[S(x^r, \xi)=\left\{
		\begin{align}
			\{i|f_i(x)-\xi>0,i\in I\}, if M(x^r,\xi) \geq 0\\
			I, if M(x^r,\xi) <0
		\end{align}\right.\]
\item 更新$\xi$
	\[\xi^{r+1} = \left\{
		\begin{align}
			\xi^r + M(x^r, \xi^r) , if M(x^r, \xi) < 0\\
			\xi^r + \lambda M(x^r, \xi^r)
		\end{align}\right.\]
\item 更新$p, p = cp, c\geq 1$,转至第2步
\end{description}
在实现中,除了计算出了最优点和最优点的函数值之外,还记录了迭代次数,函数调用次数,具体的参数设置将在结果中说明。
\subsection{Smoothing Method}
Smoothing Method是将Minimax问题转化为优化近似函数$f(x,\mu)$,其定义如下:
\[f(x,\mu) = \mu\ln\sum_{i=1}^{m}{exp(\frac{f_i(x)}{\mu})}\]
对于$f(x,\mu)$的优化使用牛顿法求出下降方向,使用Armijo准则找到步长,在每步迭代后会更新$\mu, \mu = \beta\mu$。对于$f(x,\mu)$的函数,梯度和Hessian矩阵的计算如下:
$$
\begin{aligned}
f(x^k,\mu_k)=f(x^k)+\mu_k\ln\sum_{i=1}^{m}{exp(\frac{f_i(x^k)-f(x^k)}{\mu_k})}\\
\triangledown_xf(x,\mu)=\sum_{i=1}^m{\lambda_i(x,\mu)\triangledown f_i(x)}\\
\triangledown^2_xf(x,\mu)=&\sum_{i=1}^m(\lambda_i(x,\mu)\triangledown^2f_i(x)+\frac{1}{\mu}\lambda_i(x,\mu)\triangledown f_i(x)\triangledown f_i(x)^T)\\
			& -\frac{1}{mu}(\sum_{i=1}^{m}\lambda_i(x,\mu)\triangledown f_i(x))(\sum_{i=1}^m{\lambda_i(x,\mu)\triangledown f_i(x)})^T\\
\lambda_i(x^k,\mu_k) = \frac{exp(\frac{f_i(x^k)-f(x^k)}{\mu_k})}{\sum_{j=1}^{m}{exp(\frac{f_j(x^k)-f(x^k)}{\mu_k})}}
\end{aligned}
$$
\subsection{Adaptive Smoothing Method}
Adaptive Smoothing Method是在Smoothing Method的基础对算法做了改进,主要解决了在$\mu\rightarrow 0$时产生的ill-condition问题。Adaptive Smoothing Method将Minimax问题转化为优化公式\ref{eq:adasmooth}。
\begin{equation}
\label{eq:adasmooth}
\Phi_p(x)=\log(\sum_{j\in Q}{(\exp(pf^j(x)))/p}
\end{equation}
具体算法如下:
\begin{description}
\item 取$i=0,k=0,\gamma = 1$
\item 设$\Phi_{p_i,xx}(x_i)$是Hessian矩阵R的Cholesky修正,$c(R)$为R的条件数的倒数,则若$\Phi_{p_i,xx}(x_i)$正定且$c(R)\geq k_1,p_i\leq k_3$,转步3,若$\Phi_{p_i,xx}(x_i)$正定且$c(R)\geq k_1$且$\Phi_{p_i,xx}(x_i)$的最大特征值$\sigma_{p_i,max}(x_i)$满足$\sigma_{p_i,max}\leq k_2$,转步3,否则转步4
\item $h_{p_i}(x_i) = -\Phi_{p_i,xx}(x_i)^{-1}\triangledown\Phi_{p_i}(x_i)$,转步5
\item $h_{p_i}(x_i) = -\triangledown_{p_i}(x_i)$
\item 计算步长
	\[\lambda_{p_i}(x_i) = \max_{l\in N}\{\beta^l|\Phi_{p_i}(x_i+\beta^lh_{p_i}(x_i))-\Phi_{p_i}(x_i)\leq \alpha\beta^l<\triangledown\Phi_{p_i}(x_i),h_{p_i}(x_i)>\}\]
\item $x_{i+1} = x_i + \lambda_{p_i}(x_i)h_{p_i}(x_i)$
\item 若$|\triangledown\Phi_{p_i}(x_i)|^2\leq \tao(p_i)$,转步7,否则设$p_{i+1}=p_i, i=i+1$,转步2
\item 若$p^*\leq \hat{p},\gamma=1$,其中$p^*$满足$\epsilon_a(p_i)\leq |\triangledown\Phi_{p^*}(x_i)|^2\leq \epsilon_b(p_i)$,则设$\gamma=\max\{2,(\hat{p}+2)/(k+1)\}, p_{i+1}=\gamma(k+2),k=k+1, i=i+1$,否则设$p_{i+1}=\gamma(k+2),k=k+1,i=i+1$,转步2
\end{description}
\section{实验结果及分析}
本次实验分别在Xu\cite{Xu2001}中实验中第2,3题中的函数和\cite{Charalambous1979}中的Digital filter example滤波器函数进行结果测试。实验结果如下:
\subsection{Xu\cite{Xu2001}中例子2}
问题描述如下:
\[minimize_{i=1,2,3}f_i(x)\]
其中:
\begin{align}
f_1(x) = x_1^4+x_2^2\\
f_2(x) = (2-x_1)^2 + (2-x_2)^2\\
f_3(x) = 2\exp(-x_1+x_2)
\end{align}
初始点取值为$(1,0.1)$,最优点取值为$(1,1)$。各个方法的数值结果如表\ref{tab:xu2}。
\begin{table}[htpb]
	\centering
	\caption{Xu中例子2}
	\label{tab:xu2}
	\begin{tabular}{c c c c}
	\hine
	方法 & $|f^*-f|$ & 迭代次数 & 函数调用次数 \\\hline
	Least pth & & & \\
	Smooth Method & & & \\
	Adaptive Smooth & & & \\
	\hline
	\end{tabular}
\end{table}
\subsection{Rosen-Suzuki Problem}
问题描述为:
\begin{align}
	F(x)=x_1^2+x_2^2+2x_3^2+x_4^2-5x_1-5x_2-21x_3+7x_4\\
	g_2(x)=-x_1^2-x_2^2-x_3^2-x_4^2-x_1+x_2-x_3+x_4+8\\
	g_3(x)=-x_1^2-2x_2^2-x_3^2-2x_4^2+x_1+x_4+10\\
	g_4(x)=-x_1^2-x_2^2-x_3^2-2x_1+x_2+x_4+5\\
	f_2=F(x)-\alpha_2g_2(x)\\
	f_3=F(x)-\alpha_3g_3(x)\\
	f_4=F(x)-\alpha_4g_4(x)\\
	\alpha_2=\alpha_3=\alpha_4=10
\end{align}
初始值取值为$(0,1,2,-1)$,最优解$x^*=(0,1,2,-1)$。各个方法的数值结果如表\ref{tab:rspro}。
\begin{table}[htpb]
	\centering
	\caption{Xu中例子2}
	\label{tab:rspro}
	\begin{tabular}{c c c c}
	\hine
	方法 & $|f^*-f|$ & 迭代次数 & 函数调用次数 \\\hline
	Least pth & & & \\
	Smooth Method & & & \\
	Adaptive Smooth & & & \\
	\hline
	\end{tabular}
\end{table}
\subsection{Digital filter example}
滤波器函数主要参考了\cite{}中的函数。滤波器函数定义如下:
\begin{align}
|H(\phi,\theta)|=&A\prod_{k=1}^{K}{\frac{N_k}{D_k}}\\
		&A\prod_{k=1}^{K}{(\frac{1+a_k^2+b_k^2+2b_k(2\cos^2\theta-1)+2a_k(1+b_k)\cos\theta}{1+c_k^2+d_k^2+2d_k(2\cos^2\theta-1)+2c_k(1+d_k)\cos\theta})^{1/2}}
\end{align}
其中:
\begin{align}
	\phi=[a_1,b_1,\cdots,c_K,d_k,A]^T\\
	n=4K+1
\end{align}
